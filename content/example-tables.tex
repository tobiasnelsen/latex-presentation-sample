\begin{frame}{Table example \#1}

  \begin{table}[htb]

    \centering
  
    % Instead of using the special C{Xcm} command you could use a regular "c" instead for having centered columns (try it!), 
    % but then the columns would be too wide. Giving a specific cm width prevents this.
    \begin{tabular}{ C{1.5cm}      C{0.9cm}       C{2.0cm}                  C{1.5cm}                C{2.3cm}}
                                 & \# targ. ads & Sites only Prec./Recall & Ads only Prec./Recall & Ads \& Sites Prec./Recall \\
       \toprule
                     Same cat.   & 22           & 20/10                   & 16/8                  & 17/18 \\
       \hline                       
                     Same parent & 17           & 54/34                   & 51/24                 & 53/50\\
       \hline                       
                     Same root   & 14           & 76/61                   & 82/40                 & 79/81\\
       \bottomrule
       
    \end{tabular}

    \caption{Almost a standard table (except for the fixed with of the columns)}

  \end{table}

\end{frame}

\begin{frame}{Table example \#2}

  \begin{table}[htb]
  
    \centering

    \begin{tabular}{lll}
      \toprule  
      Vertraulichkeit & Integrität &  Integritätsprüfung \\
      \midrule    

      \multirow{4.4}{*}{\shortstack[l]{AES/CTR\\(SPOC)\\(eSPOC)\\(PCoding)}} & 
      \multirow{2}{*}{HMAC} & 
      \multirow{2}{*}{E} \\

      & & \\ 
      \cmidrule(rl){2-3}

      & \multirow{2}{*}{ECDSA} & \multirow{2}{*}{E, (Z+E)} \\

      &  & \\
      \bottomrule
    \end{tabular}

    \caption{A special table in which one vertical line doesn't span all columns}

  \end{table}
  
\end{frame}