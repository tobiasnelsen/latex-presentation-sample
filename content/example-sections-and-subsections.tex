\subsection{First subsection}

\begin{frame}{Sections and subsections}

  For each \textit{\textbackslash section\{title\}} you put into your latex document, \textit{title} will appear in the \textit{table of contents} slide and in the navigation bar at the top. Take a look at the \textit{presentation.tex} file and compare! \\
  
  \vspace{0.4cm}
  
  For each slide which comes after a \textit{\textbackslash section\{title\}}, a dot will appear in the navigation bar below the corresponding \textit{title}. The dot will be filled if you visit the according slide. However, you can turn  off that a dot will be shown for individual slides if you use \textit{\textbackslash miniframesoff/on}. For example, go to the previous slide - no highlighted dot is shown for this slide!
    
  \vspace{0.4cm}
  
  You can group slides into subsections, too, by using \textit{\textbackslash subsection\{title\}}. This has the effect that the dots at the navigation bars are grouped together. Too see this effect, please take a look at which dots are highlighted while visiting the following slides!
  
\end{frame}

\subsection{Second subsection}

\begin{frame}{}

  First slide of the second subsection.

\end{frame}

\begin{frame}{}

  Second slide of the second subsection.

\end{frame}

\subsection{Third subsection}

\begin{frame}{}

  First slide of the third subsection.

\end{frame}

\begin{frame}{}

  Second slide of the third subsection.

\end{frame}